\documentclass[letterpaper]{article}


\usepackage{hyperref}
\usepackage{feynmp-auto}
\usepackage{tikz-feynman}
\usepackage{slashed}
\usepackage[vcentermath]{youngtab}
\usepackage{amsmath}
\usepackage{amssymb}
\usepackage{mathtools}
\usepackage[english]{babel}
\usepackage{url}
\usepackage{enumerate}
\usepackage{booktabs}
\usepackage{graphicx}
\usepackage{xcolor, colortbl}
\usepackage{longtable}
\usepackage{amsfonts}
\usepackage{amssymb}
\usepackage{caption}
\usepackage{multicol}
\usepackage{fancyhdr}
\usepackage{xspace}
\usepackage{braket}
\usepackage{rotating}
\usepackage{amsmath}
\usepackage{float}
\usepackage{listings}
\usepackage{color}
\usepackage{tikz} 
\usetikzlibrary{shapes,arrows,positioning,automata,backgrounds,calc,er,patterns}
\usepackage{tikz-feynman}

\numberwithin{equation}{section}
\tikzfeynmanset{compat=1.0.0}

%
\usepackage[a4paper, left=1.5cm, right=1.5cm, top=2cm, bottom=2cm]{geometry}
\setlength{\parskip}{\baselineskip}
\date{\today}

\begin{document}

\title{
  Monte Carlo Bootstrap: current developments.
  }
\author{
  A. Laio, U. Luviano, M. Serone
}

%%Cuántica
\newcommand{\inty}{\int_{-\infty}^{\infty}} 
\newcommand{\Lag}{\mathcal{L}} 
\newcommand{\Id}{\mathbb{I}} 
\newcommand{\Tr}{\mathrm{Tr}} 
\newcommand{\tr}{\mathrm{tr}} 
\newcommand{\Det}{\mathrm{Det}} 
\newcommand{\gauge}{\mathcal{A}} 
\newcommand{\Lac}{\ensuremath{\mathbf{\hat{L}^2}}} 
\newcommand{\La}{\ensuremath{\mathbf{\hat{L}}}} 
\newcommand{\Ham}{\ensuremath{\hat{H}}} 
\renewcommand{\a}{\ensuremath{(e_1)}}
\renewcommand{\b}{\ensuremath{(e_2)}}
\renewcommand{\c}{\ensuremath{(e_3)}}
\newcommand\cd{\mathrel{\stackrel{\makebox[0pt]{\mbox{\normalfont\tiny
CD}}}{\longleftrightarrow}}}
\newcommand{\tmop}[1]{\ensuremath{\operatorname{#1}}}
\newcommand{\then}{\ensuremath{\Rightarrow}}
\newcommand{\res}{\ensuremath{\mathrm{Res}}}
\newcommand{\R}{\ensuremath{\mathbb{R}}}
\newcommand{\integral}[1]{\ensuremath{\int_{a}^{b} #1 dt}}
%%%From Marco's
\def\UV{{\textsc{uv}}}
\def\IR{{\textsc{ir}}}
\def\gX{{g_\textsc{x}}}
\def\gY{{g_\textsc{y}}}
\def\dslash{\raisebox{1pt}{$\slash$} \hspace{-7pt} \partial}
\def\pslash{\raisebox{1pt}{$\slash$} \hspace{-7pt} p}
\def\qslash{\raisebox{1pt}{$\slash$} \hspace{-7pt} q}
\def\Aslash{\hspace{3pt}\raisebox{1pt}{$\slash$} \hspace{-9pt} A}
\def\Dslash{\hspace{3pt}\raisebox{1pt}{$\slash$} \hspace{-9pt} D}
\def\Dslashn{\hspace{3pt}\raisebox{1pt}{$\slash$} \hspace{-7pt} D}
\def\bea{\begin{eqnarray}} \def\eea{\end{eqnarray}}
\def\be{\begin{equation}} \def\ee{\end{equation}} \def\nn{\nonumber}
\def\a{& \hspace{-7pt}} \def\c{\hspace{-5pt}} \def\Z{{\bf Z}}
\def\zb{\overline{z}} \def\ov{\overline} \def\I{1\hspace{-4pt}1}
\def\ds{\displaystyle} \def\de{\partial} \def\deb{\ov{\partial}}
\def\dsl{\partial\hspace{-6pt}/} \def\psl{p\hspace{-5pt}/}
\def\taub{\bar \tau} \def\R{\mathcal R} \def\RR{   R}
\def\mc{\mathcal}
\def\ZZ{Z\hspace{-5pt}Z}
\def\CC{   C}
\def\b{\raisebox{14pt}{}\raisebox{-7pt}{}$\!\!$}
\def\tl{\tilde}
\def\Zeta{\ZZ}
\def\al{{\alpha^\prime}}
\def\hs{\hspace{10pt}}
\def\thp{\theta^\prime}
\def\thpt{\tilde\theta^\prime}
\def\w{\omega}
%\def\gg{{\bf g}}
\def\hh{{\bf h}}
\def\mut{\tilde \mu}
%
\newcommand{\Int}[2]{{\int}_{\hspace{-0.35cm}\begin{array}{c}\vspace{-0.35cm}
\\{\scriptstyle #1}\end{array}}\hspace{-0.35cm} #2}
\newcommand{\CInt}[2]{{\oint}_{\hspace{-0.35cm}\begin{array}{c}\vspace{-0.35cm}
\\{\scriptstyle #1}\end{array}}\hspace{-0.05cm} #2}
\newcommand{\diff}{\mathrm{d}}
\newcommand{\SO}{\text{SO}}
\newcommand{\SU}{\text{SU}}
\newcommand{\U}{\text{U}}
\newcommand{\G}{\mathcal{G}}

\newcommand{\egamma}[1]{\Gamma\!\left(#1\right)}
\newcommand{\rzeta}[1]{\zeta\!\left(#1\right)}
\setlength\arraycolsep{2pt}

\newcommand\tw[2]{
 \Bigg[\hspace{-1pt}\raisebox{1pt}
{$\begin{array}{c}
 \displaystyle{#1} \\ \displaystyle{#2}
 \end{array}$}
 \hspace{-1pt}\Bigg]}
\newcommand{\eqalign}[1]{\hspace{-10pt}\begin{array}{ll}#1\end{array}
\hspace{-10pt}}

\newcommand{\promille}{%
  \relax\ifmmode\promillezeichen
        \else\leavevmode\(\mathsurround=0pt\promillezeichen\)\fi}
\newcommand{\promillezeichen}{%
  \kern-.05em%
  \raise.5ex\hbox{\the\scriptfont0 0}%
  \kern-.15em/\kern-.15em%
  \lower.25ex\hbox{\the\scriptfont0 00}}


\newcommand{\Cr}{\star}

\newcommand{\ol}{\overline}
\newcommand{\wt}{\tilde}
\newcommand{\bs}{\boldsymbol}
\newcommand{\GR}{\mbox{\tiny $GR$ \normalsize}}
\newcommand{\abs}[1]{ \left| #1\right|}

%%%%%%%%%%%%%%%%%%%%
\maketitle

\section{Introduction}
The Conformal Bootstrap offers a way to chart the space of acceptable CFTs. 

There has recently been a great revival of interest in the conformal bootstrap program \cite{Ferrara:1973yt,Polyakov:1974gs} after ref.\cite{Rattazzi:2008pe} observed that 
its applicability extends to Conformal Field Theories (CFTs) in $d>2$ dimensions. 
Since ref.\cite{Rattazzi:2008pe}, considerable progress has been achieved in understanding CFTs in $d\geq 2$ dimensions, both numerically and analytically.
Probably the most striking progress has been made in the numerical study of the 3D Ising model,
where amazingly precise operator dimensions and OPE coefficients have been determined
\cite{ElShowk:2012ht,El-Showk:2014dwa,Kos:2016ysd}.


\section{Bootstrap Basics.}
Consider the 4-point function of a scalar primary operator $\phi$ with scaling dimension $\Delta_\phi$: 
\begin{equation}
\label{4-point-function}
\langle \phi(x_1) \phi(x_2) \phi(x_3) \phi(x_4) \rangle \, = \, \frac{g(z,\bar z)}{x_{12}^{2\Delta_\phi}x_{34}^{2\Delta_\phi} } \,,
\end{equation}
where $z$ and $\bar z$ are so called conformally-invariant cross-ratios made up of $x_{ij}\equiv x_i-x_j$.
Applying the OPE to the operator pairs $\phi(x_1) \phi(x_2)$ and $\phi(x_3) \phi(x_4)$ in the 4-point function, one can write
\begin{equation}
g(z,\bar z) = 1+\sum_{\Delta ,l } p_{\mathcal{O}} \, g_{\Delta,l}(z,\bar z) \,, 
\label{CBExp}
 \end{equation}
where the sum runs over all primary operators $\mathcal{O}$ that appear in the $\phi\times \phi$ OPE with  $\Delta$ and $l$ being respectively their dimension and spin. Notice that $l$ has to be even
for identical scalars. For each primary, the sum over all its descendants is encoded in the conformal block function $g_{\Delta,l}(z,\bar z)$. Reflection positivity (i.e. unitarity) implies bounds on both the squared OPE coefficients $p_{\mathcal{O}}$
and the scaling dimensions $\Delta$:
\begin{equation}
p_{\mathcal{O}} >0 \;\; \forall \mathcal{O}\,,  \quad
 \Delta \geq l+d-2 \;\; (l>0)\,, \quad \Delta \geq \frac{d-2}2 \;\; (l=0).
  \label{unit-bound}
\end{equation}

Imposing crossing symmetry to the 4-point function (\ref{4-point-function}) gives, in any number of dimensions, 
\begin{align}
\label{bootstrapequation}
\sum_{\Delta ,l } p_{\mathcal{O}} \, \mathcal{F}_{\mathcal{O}}(z, \bar z) \, = \, (z \bar z)^{\Delta_\phi} - ((1-z)(1-\bar z))^{\Delta_\phi} \, ,
\end{align} 
where
\begin{equation}
\mathcal{F}_{\mathcal{O}}(z, \bar z) \, \equiv \, ((1-z)(1-\bar z))^{\Delta_\phi} g_{\Delta,l}(z,\bar z)- (z \bar z)^{\Delta_\phi} g_{\Delta,l}(1-z,1-\bar z) \,.
\end{equation}


\section{General Intuition of the MC bootstrap}
One can consider \ref{bootstrapequation} as a function of the spectrum and the
OPE coefficients, which should only be satisfied for physical conformal data.
Since any numerical approach involves truncating the OPE up to a maximum
dimension $\Delta^*$, it is important to estimate the remainder of the series in
order to bound the amount by which \ref{bootstrapequation} might be violated in
a given approximation. This was done by Rychkov et al,
who found the following error estimate holds (for $\Delta_*\gg 1$):
\begin{equation}
\Big|\sum_{(\Delta\geq \Delta_*),l}p_{\mathcal{O}} \, g_{\Delta,l}(z,\bar z)\Big| \, \leq \, \mathcal{R}(z,\bar z)\,,
\label{finalest4}
\end{equation}
where
\begin{equation}
\mathcal{R}(z,\bar z) \,\equiv \, \frac{(-\log|\rho(z)|)^{-4\Delta_\phi+1} \, 2^{4\Delta_\phi+1}}{\Gamma(4\Delta_\phi)} \frac{\Gamma(4\Delta_\phi, - \Delta_* \log |\rho(z)|)}{{(1-|\rho(z)|^2)}}\,.
\label{finalest5a}
\end{equation}

This motivates trying to find conformal data numerically using some sort of
gradient descent. In this case, one can take the truncated 
Bootstrap equations as a function of a discrete spectrum and OPE coefficients 
and minimize it with a Monte Carlo
simulation. Concretely, one can rewrite the bootstrap equations as 

\begin{equation}
\label{bootstrapequation-bound}
\sum_{\Delta<\Delta^* ,l } p_{\mathcal{\Delta},l} \,
  \mathcal{F}_{\mathcal{O}}(z, \bar z) \, = \, (z \bar z)^{\Delta_\phi} -
  ((1-z)(1-\bar z))^{\Delta_\phi} + \mathcal{E}(z,\bar z),
\end{equation} 

where $l$ is bounded in turn by
  \ref{unit-bound}, and 
\begin{equation}
\label{error-bound}
  \mathcal{E}(z,\bar z) \leq \mathcal{E}_{\mathrm{max}}(z,\bar z) \equiv
  v^{\Delta_\phi}\mathcal{R}(z,\bar z)
  +
  u^{\Delta_\phi}\mathcal{R}(1-z,1-\bar z).
\end{equation} 

Since these equations have to be satisfied everywhere outside the branch cut at
\([1,\infty)\), one can take a finite set of points $\{z_i\}_{i=1}^{N_z}$ and
write \ref{bootstrapequation-bound} as
\begin{equation}
  \mathcal{M}\cdot \vec{\rho} = \vec{\sigma} + \vec{\epsilon}
\label{matrix-boo}
\end{equation}

where 
\begin{equation}
\vec{\sigma} \, \equiv \, \left( \begin{matrix}
                      |z_1|^{2\Delta_\phi} \, - \, |1-z_1|^{2 \Delta_\phi}\\  |z_2|^{2\Delta_\phi} \, - \, |1-z_2|^{2 \Delta_\phi} \\ \vdots
                     \end{matrix} \right) 
                     \quad \, \text{and} \, \quad 
                      \vec{\epsilon} \, \equiv \, 
                          \left( \begin{matrix}
                                  \mathcal{E}(z_1,\bar z_1,\Delta_*,\Delta_\phi) \\
                                  \mathcal{E}(z_2,\bar z_2,\Delta_*,\Delta_\phi) \\
                                  \vdots
                                 \end{matrix} \right)  \, .
\end{equation}

and
\begin{equation}
\mathcal{M} \, \equiv \, \left( \begin{matrix}
  \mathcal{F}_{\Delta_1,l_1}(z_1, \bar z_1) &\mathcal{F}_{\Delta_2,l_2}(z_1, \bar
  z_1) & \cdots \\
  \mathcal{F}_{\Delta_1,l_1}(z_2, \bar z_2) &\mathcal{F}_{\Delta_2,l_2}(z_2, \bar
  z_2) & \cdots \\
  \vdots  & \vdots &\ddots\\
                     \end{matrix} \right) .
                     \label{matrix-op}
\end{equation}



The straightforward path is to define a cost function
\[
  C\sim | \mathcal{M}\cdot \vec{\rho} - \vec{\sigma} |
\]
and minimize it stochastically.
This naive approach proved to be untractable in the early stages of the
research, mostly because it became exponentially time consuming to look for more
operators. Thus, we decided to split the optimization in two steps. One that
finds an approximate spectrum without worrying about the OPE coefficients, and
then solving for the complete conformal data with a more refined Monte Carlo.
The first step was motivated by Gliozzi's determinant bootstrap, which we review
in the following section.

\section{Review of Gliozzi's paper}
yadda yadda matrix, solution

\section{LogDet implementation}
Given a putative spectrum, the determinant of \ref{matrix-op}
should get closer to zero as we approach a physically sensible set of operator
dimensions. Since the determinant
\subsection{naive approach}
\subsection{Cross smearing}
\subsection{Recursive with check}

\section{$\chi^2$ refinement}
\subsection{cWLS as a way to determine OPEs}

\section{Summary of results}
\subsection{Free CFT in 4d}
\subsection{GFT in 1d}
\subsection{GFT in 4d}

  \end{document}
