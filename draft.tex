\documentclass[letterpaper]{article}


\usepackage{hyperref}
\usepackage{feynmp-auto}
\usepackage{tikz-feynman}
\usepackage{slashed}
\usepackage[vcentermath]{youngtab}
\usepackage{amsmath}
\usepackage{amssymb}
\usepackage{mathtools}
\usepackage[english]{babel}
\usepackage{url}
\usepackage{enumerate}
\usepackage{booktabs}
\usepackage{graphicx}
\usepackage{xcolor, colortbl}
\usepackage{longtable}
\usepackage{amsfonts}
\usepackage{amssymb}
\usepackage{caption}
\usepackage{multicol}
\usepackage{fancyhdr}
\usepackage{xspace}
\usepackage{braket}
\usepackage{rotating}
\usepackage{amsmath}
\usepackage{float}
\usepackage{listings}
\usepackage{color}
\usepackage{tikz} 
\usetikzlibrary{shapes,arrows,positioning,automata,backgrounds,calc,er,patterns}
\usepackage{tikz-feynman}

\numberwithin{equation}{section}
\tikzfeynmanset{compat=1.0.0}

%
\usepackage[a4paper, left=1.5cm, right=1.5cm, top=2cm, bottom=2cm]{geometry}
\setlength{\parskip}{\baselineskip}
\date{\today}

\begin{document}

\title{
  Monte Carlo Bootstrap: current developments.
  }
\author{
  A. Laio, U. Luviano, M. Serone
}

%%Cuántica
\newcommand{\inty}{\int_{-\infty}^{\infty}} 
\newcommand{\Lag}{\mathcal{L}} 
\newcommand{\Id}{\mathbb{I}} 
\newcommand{\Tr}{\mathrm{Tr}} 
\newcommand{\tr}{\mathrm{tr}} 
\newcommand{\Det}{\mathrm{Det}} 
\newcommand{\gauge}{\mathcal{A}} 
\newcommand{\Lac}{\ensuremath{\mathbf{\hat{L}^2}}} 
\newcommand{\La}{\ensuremath{\mathbf{\hat{L}}}} 
\newcommand{\Ham}{\ensuremath{\hat{H}}} 
\renewcommand{\a}{\ensuremath{(e_1)}}
\renewcommand{\b}{\ensuremath{(e_2)}}
\renewcommand{\c}{\ensuremath{(e_3)}}
\newcommand\cd{\mathrel{\stackrel{\makebox[0pt]{\mbox{\normalfont\tiny
CD}}}{\longleftrightarrow}}}
\newcommand{\tmop}[1]{\ensuremath{\operatorname{#1}}}
\newcommand{\then}{\ensuremath{\Rightarrow}}
\newcommand{\res}{\ensuremath{\mathrm{Res}}}
\newcommand{\R}{\ensuremath{\mathbb{R}}}
\newcommand{\integral}[1]{\ensuremath{\int_{a}^{b} #1 dt}}
%%%From Marco's
\def\UV{{\textsc{uv}}}
\def\IR{{\textsc{ir}}}
\def\gX{{g_\textsc{x}}}
\def\gY{{g_\textsc{y}}}
\def\dslash{\raisebox{1pt}{$\slash$} \hspace{-7pt} \partial}
\def\pslash{\raisebox{1pt}{$\slash$} \hspace{-7pt} p}
\def\qslash{\raisebox{1pt}{$\slash$} \hspace{-7pt} q}
\def\Aslash{\hspace{3pt}\raisebox{1pt}{$\slash$} \hspace{-9pt} A}
\def\Dslash{\hspace{3pt}\raisebox{1pt}{$\slash$} \hspace{-9pt} D}
\def\Dslashn{\hspace{3pt}\raisebox{1pt}{$\slash$} \hspace{-7pt} D}
\def\bea{\begin{eqnarray}} \def\eea{\end{eqnarray}}
\def\be{\begin{equation}} \def\ee{\end{equation}} \def\nn{\nonumber}
\def\a{& \hspace{-7pt}} \def\c{\hspace{-5pt}} \def\Z{{\bf Z}}
\def\zb{\overline{z}} \def\ov{\overline} \def\I{1\hspace{-4pt}1}
\def\ds{\displaystyle} \def\de{\partial} \def\deb{\ov{\partial}}
\def\dsl{\partial\hspace{-6pt}/} \def\psl{p\hspace{-5pt}/}
\def\taub{\bar \tau} \def\R{\mathcal R} \def\RR{   R}
\def\mc{\mathcal}
\def\ZZ{Z\hspace{-5pt}Z}
\def\CC{   C}
\def\b{\raisebox{14pt}{}\raisebox{-7pt}{}$\!\!$}
\def\tl{\tilde}
\def\Zeta{\ZZ}
\def\al{{\alpha^\prime}}
\def\hs{\hspace{10pt}}
\def\thp{\theta^\prime}
\def\thpt{\tilde\theta^\prime}
\def\w{\omega}
%\def\gg{{\bf g}}
\def\hh{{\bf h}}
\def\mut{\tilde \mu}
%
\newcommand{\Int}[2]{{\int}_{\hspace{-0.35cm}\begin{array}{c}\vspace{-0.35cm}
\\{\scriptstyle #1}\end{array}}\hspace{-0.35cm} #2}
\newcommand{\CInt}[2]{{\oint}_{\hspace{-0.35cm}\begin{array}{c}\vspace{-0.35cm}
\\{\scriptstyle #1}\end{array}}\hspace{-0.05cm} #2}
\newcommand{\diff}{\mathrm{d}}
\newcommand{\SO}{\text{SO}}
\newcommand{\SU}{\text{SU}}
\newcommand{\U}{\text{U}}
\newcommand{\G}{\mathcal{G}}

\newcommand{\egamma}[1]{\Gamma\!\left(#1\right)}
\newcommand{\rzeta}[1]{\zeta\!\left(#1\right)}
\setlength\arraycolsep{2pt}

\newcommand\tw[2]{
 \Bigg[\hspace{-1pt}\raisebox{1pt}
{$\begin{array}{c}
 \displaystyle{#1} \\ \displaystyle{#2}
 \end{array}$}
 \hspace{-1pt}\Bigg]}
\newcommand{\eqalign}[1]{\hspace{-10pt}\begin{array}{ll}#1\end{array}
\hspace{-10pt}}

\newcommand{\promille}{%
  \relax\ifmmode\promillezeichen
        \else\leavevmode\(\mathsurround=0pt\promillezeichen\)\fi}
\newcommand{\promillezeichen}{%
  \kern-.05em%
  \raise.5ex\hbox{\the\scriptfont0 0}%
  \kern-.15em/\kern-.15em%
  \lower.25ex\hbox{\the\scriptfont0 00}}


\newcommand{\Cr}{\star}

\newcommand{\ol}{\overline}
\newcommand{\wt}{\tilde}
\newcommand{\bs}{\boldsymbol}
\newcommand{\GR}{\mbox{\tiny $GR$ \normalsize}}
\newcommand{\abs}[1]{ \left| #1\right|}

%%%%%%%%%%%%%%%%%%%%
\maketitle

\section{Introduction}
The Conformal Bootstrap offers a way to chart the space of unitary and crossing
symmetric CFTs. 

There has recently been a great revival of interest in the conformal bootstrap program \cite{Ferrara:1973yt,Polyakov:1974gs} after ref.\cite{Rattazzi:2008pe} observed that 
its applicability extends to Conformal Field Theories (CFTs) in $d>2$ dimensions. 
Since ref.\cite{Rattazzi:2008pe}, considerable progress has been achieved in understanding CFTs in $d\geq 2$ dimensions, both numerically and analytically.
Probably the most striking progress has been made in the numerical study of the 3D Ising model,
where amazingly precise operator dimensions and OPE coefficients have been determined
\cite{ElShowk:2012ht,El-Showk:2014dwa,Kos:2016ysd}.


\section{Bootstrap Basics.}
Consider the 4-point function of a scalar primary operator $\phi$ with scaling dimension $\Delta_\phi$: 
\begin{equation}
\label{4-point-function}
\langle \phi(x_1) \phi(x_2) \phi(x_3) \phi(x_4) \rangle \, = \, \frac{g(z,\bar z)}{x_{12}^{2\Delta_\phi}x_{34}^{2\Delta_\phi} } \,,
\end{equation}
where $z$ and $\bar z$ are so called conformally-invariant cross-ratios made up of $x_{ij}\equiv x_i-x_j$.
Applying the OPE to the operator pairs $\phi(x_1) \phi(x_2)$ and $\phi(x_3) \phi(x_4)$ in the 4-point function, one can write
\begin{equation}
g(z,\bar z) = 1+\sum_{\Delta ,l } p_{\mathcal{O}} \, g_{\Delta,l}(z,\bar z) \,, 
\label{CBExp}
 \end{equation}
where the sum runs over all primary operators $\mathcal{O}$ that appear in the $\phi\times \phi$ OPE with  $\Delta$ and $l$ being respectively their dimension and spin. Notice that $l$ has to be even
for identical scalars. For each primary, the sum over all its descendants is encoded in the conformal block function $g_{\Delta,l}(z,\bar z)$. Reflection positivity (i.e. unitarity) implies bounds on both the squared OPE coefficients $p_{\mathcal{O}}$
and the scaling dimensions $\Delta$:
\begin{equation}
p_{\mathcal{O}} >0 \;\; \forall \mathcal{O}\,,  \quad
 \Delta \geq l+d-2 \;\; (l>0)\,, \quad \Delta \geq \frac{d-2}2 \;\; (l=0).
  \label{unit-bound}
\end{equation}

Imposing crossing symmetry to the 4-point function (\ref{4-point-function}) gives, in any number of dimensions, 
\begin{align}
\label{bootstrapequation}
\sum_{\Delta ,l } p_{\mathcal{O}} \, \mathcal{F}_{\mathcal{O}}(z, \bar z) \, = \, (z \bar z)^{\Delta_\phi} - ((1-z)(1-\bar z))^{\Delta_\phi} \, ,
\end{align} 
where
\begin{equation}
\mathcal{F}_{\mathcal{O}}(z, \bar z) \, \equiv \, ((1-z)(1-\bar z))^{\Delta_\phi} g_{\Delta,l}(z,\bar z)- (z \bar z)^{\Delta_\phi} g_{\Delta,l}(1-z,1-\bar z) \,.
\end{equation}


\section{General Intuition of the MC bootstrap}
One can consider \ref{bootstrapequation} as a function of the spectrum and the
OPE coefficients, which should only be satisfied for physical conformal data.
Since almost any numerical approach involves truncating the OPE up to a maximum
dimension $\Delta^*$, it is important to estimate the remainder of the series in
order to bound the amount by which \ref{bootstrapequation} might be violated in
a given approximation. This was done by \cite{Pappadopulo:2012jk},
who found the following error estimate holds (for $\Delta_*\gg 1$):
\begin{equation}
\Big|\sum_{(\Delta\geq \Delta_*),l}p_{\mathcal{O}} \, g_{\Delta,l}(z,\bar z)\Big| \, \leq \, \mathcal{R}(z,\bar z)\,,
\label{finalest4}
\end{equation}
where
\begin{equation}
\mathcal{R}(z,\bar z) \,\equiv \, \frac{(-\log|\rho(z)|)^{-4\Delta_\phi+1} \, 2^{4\Delta_\phi+1}}{\Gamma(4\Delta_\phi)} \frac{\Gamma(4\Delta_\phi, - \Delta_* \log |\rho(z)|)}{{(1-|\rho(z)|^2)}}\,.
\label{finalest5a}
\end{equation}

This motivates trying to find conformal data numerically using some sort of
gradient descent. In this case, one can take the truncated 
Bootstrap equations as a function of a discrete spectrum and OPE coefficients 
and minimize it with a Monte Carlo
simulation. Concretely, one can rewrite the bootstrap equations as 

\begin{equation}
\label{bootstrapequation-bound}
\sum_{\Delta<\Delta^* ,l } p_{\mathcal{\Delta},l} \,
  \mathcal{F}_{\mathcal{O}}(z, \bar z) \, = \, (z \bar z)^{\Delta_\phi} -
  ((1-z)(1-\bar z))^{\Delta_\phi} + \mathcal{E}(z,\bar z),
\end{equation} 

where $l$ is bounded in turn by
  \ref{unit-bound}, and 
\begin{equation}
\label{error-bound}
  |\mathcal{E}(z,\bar z)| \leq \mathcal{E}_{\mathrm{max}}(z,\bar z) \equiv
  v^{\Delta_\phi}\mathcal{R}(z,\bar z)
  +
  u^{\Delta_\phi}\mathcal{R}(1-z,1-\bar z).
\end{equation} 

Since these equations have to be satisfied everywhere outside the branch cut at
\([1,\infty)\), one can take a finite set of points $\{z_i\}_{i=1}^{N_z}$ and
write \ref{bootstrapequation-bound} as
\begin{equation}
  \mathcal{M}\cdot \vec{\rho} = \vec{\sigma} + \vec{\epsilon}
\label{matrix-boo}
\end{equation}

where 
\begin{equation}
\vec{\sigma} \, \equiv \, \left( \begin{matrix}
                      |z_1|^{2\Delta_\phi} \, - \, |1-z_1|^{2 \Delta_\phi}\\  |z_2|^{2\Delta_\phi} \, - \, |1-z_2|^{2 \Delta_\phi} \\ \vdots
                     \end{matrix} \right) 
                     \quad \, , \, \quad 
                      \vec{\epsilon} \, \equiv \, 
                          \left( \begin{matrix}
                                  \mathcal{E}(z_1,\bar z_1,\Delta_*,\Delta_\phi) \\
                                  \mathcal{E}(z_2,\bar z_2,\Delta_*,\Delta_\phi) \\
                                  \vdots
                                 \end{matrix} \right)  \, ,
\end{equation}

and
\begin{equation}
\mathcal{M} \, \equiv \, \left( \begin{matrix}
  \mathcal{F}_{\Delta_1,l_1}(z_1, \bar z_1) &\mathcal{F}_{\Delta_2,l_2}(z_1, \bar
  z_1) & \cdots \\
  \mathcal{F}_{\Delta_1,l_1}(z_2, \bar z_2) &\mathcal{F}_{\Delta_2,l_2}(z_2, \bar
  z_2) & \cdots \\
  \vdots  & \vdots &\ddots\\
                     \end{matrix} \right) .
                     \label{matrix-op}
\end{equation}



The straightforward path would be to define a cost function
\[
  C(\Delta_i,\vec\rho)\sim | \mathcal{M}(\Delta_i)\cdot \vec{\rho} - \vec{\sigma} |
\]
and minimize it stochastically over the space of $\{\Delta_i,\vec\rho\}$ \footnote{Should I add a section on Metropolis MC
here?}.
This naive approach proved to be untractable in the early stages of the
research, mostly because it became exponentially time consuming to look for more
operators. Thus, we decided to split the optimization into two steps. First, we
find an approximate spectrum without worrying about the OPE coefficients, and
then solve for the complete conformal data with a more refined Monte Carlo.
The first step was motivated by Gliozzi's determinant bootstrap, which we review
in the following section.

\section{Review of Gliozzi's paper}
In 2013, \cite{Gliozzi:2013ysa} realized that, for $N_z$ bigger than the number of operators
one considered in the OPE ($N_{Op}$), \ref{matrix-boo} represented an over-determined
Linear System. Thus, in order for a solution (that is, a set of OPE coefficients
satisfying crossing) to exist, each minor should vanish.

\section{LogDet implementation}
\subsection{naive approach}
Given a putative spectrum, the determinant of \ref{matrix-op} (when
$Nz=N_{Op}+1$)
should get closer to zero as we approach a physically sensible set of operator
dimensions. Since this quantity could range over many orders of magnitude, we
decided to square it and then take the $\log$ of it. This proved to be
numerically troubling due to the behaviour of the logarithm around $0$  and the landscape was thus full of spurious minima out of
which the Metropolis algorithm could not get out.

\subsection{Cross smearing}
In order to smooth this behaviour out, we implemented an empirical
renormalization that consists in computing the value of the determinant at
adjacent points to the main one. Thus
\[
\begin{aligned}
  S(\{\Delta_i,l_i\})& = \log\Det^2\mathcal{M}(\{\Delta_i,l_i\})\\
  \to
  S_{\mathrm{cross}}(\{\Delta_i,l_i\})  & =   \log\Det^2\mathcal{M}(\{\Delta_i,l_i\}) + \sum_i\left(
  \log\Det^2\mathcal{M}(\cdots,\{\Delta_i+\delta,l_i\},\cdots)  +
  \log\Det^2\mathcal{M}(\cdots,\{\Delta_i-\delta,l_i\},\cdots)  \right)
\end{aligned}
\]

This action was much more efficient at finding the correct dimensions for the
free theory, and numerical exploration showed that the basin of attraction for
each given operator's dimension was very sharp, as long as the other ones were
kept at the exact value. Small perturbations destroyed this basin rather
quickly, and thus it was hard to get the MC routine to solve for a large number
of operators  ($\sim 7$) at the same time, given generic initial conditions.

\subsection{Recursive with check}
\label{rec-check}

The next breakthrough in the routine came when we decided to start with few
($\sim 4$) operators, find the operator dimensions which minimized the 
action and then add the next operators one by one. To avoid spoiling
the previous solution, only the new operator was varied for a given number of
steps before we allowed the whole spectrum to be optimized. This gave a very
efficient routine with a large radius of convergence. (As of 19/02, we have
convergence for 50\% above exact dimension for $l=0,2,4,6$ and 10\% for $8\leq l
\leq 48$, using 3000 iterations for the first step and 1500 for each subsequent
one).

For unitary theories, this recursive step was further improved by using
\ref{error-bound} to bound \ref{matrix-boo} as 
\begin{equation}
  | \mathcal{M}\cdot \vec{\rho}-\vec{\sigma} | < \vec{\epsilon}_{\mathrm{max}},
  \label{cross-check}
\end{equation}
where $\vec \epsilon _{\mathrm{max}}$ is just $\vec \epsilon$ under the
substitution $\mathcal{E}(z,\bar z;
\Delta^*)\to\mathcal{E}_{\mathrm{max}}(z,\bar z; \Delta^*)$.
Then, the result of each end state of the MC could be tested to see if it
satisfied Crossing Symmetry with the precision correspoinding to the $\Delta^*$
considered\footnote{$\vec \rho$ was determined by a modified $\chi^2$
minimization program. See \ref{cWLS}.}. Otherwise, it changed the seed of the pseudorandom generator and
started again the MC. The routine ended if at a given number of operators, more
than a certain number of repetitions was achieved, thus avoiding waste of CPU
time. To avoid considering solutions with smaller $\Delta^*$ at each iteration,
we added the explicit condition of letting the number of zero entries in $\vec
\rho$ increase at most in one from run to run. This routine managed to get
relatively good results up to $l=18$, where it became increasingly difficult to
satisfy the crossing equations.


\section{$\chi^2$ refinement}
\subsection{cWLS as a way to determine the OPE coefficients}
\label{cWLS}
Since the dependence of \ref{cross-check} on $\vec{\rho}$ is linear, for
$N_z>N_{Op}$ it is possible to find the set of OPE coefficients that best
satisfy crossing via the Weighted Least Squares method\footnote{Should we add a
reference here, other than Wikipedia?}. 
For a generic spectrum
obtained via the method described in section \ref{rec-check}, this naive set of OPE
coefficients will have some negative components, which stems from the
fact that our spectrum is only approximate. Since the error estimate
\label{finalest4}
depends
crucially on the positivity of the OPE coefficients, we need to find the
solution that minimizes \ref{cross-check} and has all entries non-negative. This
is attained by setting the first negative OPE coefficient (from back to forth,
if $\Delta_i<\Delta_{i+1}$) and then running a WLS routine with the remaining
entries. If this solution still has negative entries, the proccedure is repeated
until an acceptable $\vec{\rho}$ is obtained.


\subsection{Number of Violations (NoV) minimization}
It is an interesting fact observed in the cases studied so far, that this
constrained solution doesn't satisfy crossing with the error estimate
corresponding to the original $\Delta^*$, but once we consider the zeros in
$\vec{\rho}$, which effectively remove higher dimensional operators from the
OPE, thus reducing $\Delta^*$, \ref{cross-check} is satisfied.

For properly chosen
points in $z$, not all the inequalties in \ref{cross-check} will be satisfied,
one can then take this number and try to minimize it with another Monte Carlo
routine. Concretely, this routine takes the naive $\Delta^*$ and given a certain
number of $z$-points, it counts for how many of these points are the crossing
equations are violated (NoV). It also counts how many zero coefficients are there in
the solution. Then, the optimization proceeds by changing one of the dimensions
whose corresponding OPE coefficients are not zero and then calculating the NoV
and the number of zero coefficients again. If both the NoV and the number of
zero OPE coefficients remain constant or decrease, the move is accepted,
otherwise it is rejected. The optimization terminates when NoV=0 for two
different sets of z points or when a certain number of iterations is attained.

\section{State of the art}
The best routine we have so far starts with 4 operators, uses the logdet routine
to find an approximate set of dimensions and then takes this as a starting
condition for a NoV minimization, searching to refine the solution and obtain a
higher number of non-zero OPE coefficients. To this solution a higher operator
is appended and this new spectrum is taken as the initial condition of another
logdet MC. This continues as long as the solution satisfies crossing at every
point (using the effective $\Delta^*$) and each successive step does not add
more than one extra zero OPE coefficient. I.e.
$\mathrm{zeros}(\{\Delta_l'\})\leq \mathrm{zeros}(\{\Delta_l\})+1$.

\section{Summary of results}
\subsection{Free CFT in 4d}
\subsection{GFT in 1d}
\subsection{GFT in 4d}

\begin{thebibliography}{10}

 
 \bibitem{Ferrara:1973yt}
  S.~Ferrara, A.~F.~Grillo and R.~Gatto,
  ``Tensor representations of conformal algebra and conformally covariant operator product expansion,''
  Annals Phys.\  {\bf 76} (1973) 161.
  %%CITATION = APNYA,76,161;%%
  %88 citations counted in INSPIRE as of 18 Nov 2014
  %  \bibitem{Ferrara:1973eg}
  
  \bibitem{Polyakov:1974gs}
  A.~M.~Polyakov,
  ``Nonhamiltonian approach to conformal quantum field theory,''
  Zh.\ Eksp.\ Teor.\ Fiz.\  {\bf 66} (1974) 23.
  %%CITATION = ZETFA,66,23;%%
  %67 citations counted in INSPIRE as of 10 Jun 2014


\bibitem{Rattazzi:2008pe}
  R.~Rattazzi, V.~S.~Rychkov, E.~Tonni and A.~Vichi,
  ``Bounding scalar operator dimensions in 4D CFT,''
  JHEP {\bf 0812} (2008) 031
  [arXiv:0807.0004 [hep-th]].
  %%CITATION = ARXIV:0807.0004;%%
  %171 citations counted in INSPIRE as of 13 May 2015

 \bibitem{ElShowk:2012ht}
  S.~El-Showk, M.~F.~Paulos, D.~Poland, S.~Rychkov, D.~Simmons-Duffin and A.~Vichi,
  ``Solving the 3D Ising Model with the Conformal Bootstrap,''
  Phys.\ Rev.\ D {\bf 86} (2012) 025022
  doi:10.1103/PhysRevD.86.025022
  [arXiv:1203.6064 [hep-th]].
  %%CITATION = doi:10.1103/PhysRevD.86.025022;%%
  %156 citations counted in INSPIRE as of 13 Apr 2016

    \bibitem{El-Showk:2014dwa}
  S.~El-Showk, M.~F.~Paulos, D.~Poland, S.~Rychkov, D.~Simmons-Duffin and A.~Vichi,
  ``Solving the 3d Ising Model with the Conformal Bootstrap II. c-Minimization and Precise Critical Exponents,''
  J.\ Stat.\ Phys.\  {\bf 157} (2014) 869
  doi:10.1007/s10955-014-1042-7
  [arXiv:1403.4545 [hep-th]].
  %%CITATION = doi:10.1007/s10955-014-1042-7;%%
  %95 citations counted in INSPIRE as of 13 Apr 2016
  
   \bibitem{Kos:2016ysd}
  F.~Kos, D.~Poland, D.~Simmons-Duffin and A.~Vichi,
  ``Precision Islands in the Ising and $O(n)$ Models,''
  arXiv:1603.04436 [hep-th].
  %%CITATION = ARXIV:1603.04436;%%
  %3 citations counted in INSPIRE as of 13 Apr 2016 
  
  \bibitem{Kos:2013tga}
  F.~Kos, D.~Poland and D.~Simmons-Duffin,
  ``Bootstrapping the $O(n)$ vector models,''
  JHEP {\bf 1406} (2014) 091
  doi:10.1007/JHEP06(2014)091
  [arXiv:1307.6856 [hep-th]].
  %%CITATION = doi:10.1007/JHEP06(2014)091;%%
  %85 citations counted in INSPIRE as of 13 Apr 2016

%\cite{Gliozzi:2013ysa}
\bibitem{Gliozzi:2013ysa}
  F.~Gliozzi,
  ``More constraining conformal bootstrap,''
  Phys.\ Rev.\ Lett.\  {\bf 111} (2013) 161602
  doi:10.1103/PhysRevLett.111.161602
  [arXiv:1307.3111 [hep-th]].
  %%CITATION = doi:10.1103/PhysRevLett.111.161602;%%
  %101 citations counted in INSPIRE as of 05 Feb 2019

  %\cite{Pappadopulo:2012jk}
\bibitem{Pappadopulo:2012jk}
  D.~Pappadopulo, S.~Rychkov, J.~Espin and R.~Rattazzi,
  ``OPE Convergence in Conformal Field Theory,''
  Phys.\ Rev.\ D {\bf 86} (2012) 105043
  doi:10.1103/PhysRevD.86.105043
  [arXiv:1208.6449 [hep-th]].
  %%CITATION = doi:10.1103/PhysRevD.86.105043;%%
  %152 citations counted in INSPIRE as of 05 Feb 2019


\end{thebibliography}

  \end{document}
